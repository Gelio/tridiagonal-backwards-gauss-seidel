\documentclass[12pt]{article}
\usepackage{polski}
\usepackage[utf8]{inputenc}
\usepackage{amsfonts}
\usepackage{amsmath}
\setlength{\parskip}{1em}


\begin{document}
	\title{Sprawozdanie\\Metody Numeryczne 2, laboratorium 1}
	\author{Grzegorz Rozdzialik (D4, grupa lab. 2)}
	\maketitle	

	\section{Zadanie}
	Rozwiązywanie układu równań z macierzą trójdiagonalną w dziedzinie zespolonej metodą Gaussa-Seidla w tył (Backwards Gauss-Seidel).
	
	\section{Opis metody}
	Metoda Gaussa-Seidla w tył jest iteracyjną metodą rozwiązywania układów równań liniowych. Jest ona podobna do zwykłej metody Gaussa-Seidla, przy czym w tym przypadku obliczenia rozpoczynamy od ostatnich elementów wektora rozwiązań.
	
	Szukamy rozwiązania układu równań $Ax = b$, gdzie
	$A \in \mathbb{C}^{n \times n}$ ($A$ - macierz trójdiagonalna),
	$b \in \mathbb{C}^{n}$,
	$n \in \mathbb{N}$.
	
	Mając przybliżenie początkowe $x^{(0)} \in \mathbb{C}^n$ tworzymy ciąg kolejnych przybliżeń rozwiązań $\left\{ x^{(k)} \right\} (k = 1, 2, \dots)$ taki, że
	$$\lim_{k \to +\infty} x^{(k)} = x$$

	Układ $Ax = b$ można zapisać w postaci:
	\[
	\begin{bmatrix}
		d_1    & u_1    & 0            & 0      & 0       &         & \dots   &         & 0       \\
		l_2    & d_2    & u_2          & 0      & 0       &         & \dots   &         & 0       \\
		0      & l_3    & d_3          & u_3    & 0       &         & \dots   &         & 0       \\
		0      & 0      & l_4          & d_4    & u_4     &         & \dots   &         & 0       \\
		       &        &              & \ddots & \ddots  & \ddots  &         &         & \vdots  \\
		\vdots & \vdots &              & 0      & l_{n-3} & d_{n-3} & u_{n-3} & 0       & 0       \\
		       &        &              &        & 0       & l_{n-2} & d_{n-2} & u_{n-2} & 0       \\
		       &        &              &        &         & 0       & l_{n-1} & d_{n-1} & u_{n-1} \\
		0      & 0      & \hdotsfor{3} &        & 0       & l_n     & d_n     &
	\end{bmatrix}
	\begin{bmatrix}
		x_1 \\
		x_2 \\
		x_3 \\
		\\
		\vdots\\
		\\
		\\
		\\
		x_{n-1} \\
		x_n
	\end{bmatrix}
	=
	\begin{bmatrix}
		b_1 \\
		b_2 \\
		b_3 \\
		\\
		\vdots\\
		\\
		\\
		\\
		b_{n-1} \\
		b_n
	\end{bmatrix}
	\]
	gdzie $l_i, d_i, u_i, b_i, x_i \in \mathbb{C}^n$ dla każdego $i=1, \dots , n$.
	
	Po wymnożeniu otrzymujemy następujący układ równań:
	\[
	\begin{cases}
		\begin{aligned}
			d_1 x_1 + u_1 x_2                               & = b_1     \\
			l_2 x_1 + d_2 x_2 + u_2 x_3                     & = b_2     \\
			l_3 x_2 + d_3 x_3 + u_3 x_4                     & = b_3     \\
			\vdots    \hspace{1em}                          &  \\
			l_i x_{i-1} + d_i x_i + u_i x_{i+1}             & = b_i     \\
			\vdots        \hspace{1em}                      &  \\
			l_{n-1} x_{n-2} + d_{n-1} x_{n-1} + u_{n-1} x_n & = b_{n-1} \\
			l_n x_{n-1} + d_n x_n                           & = b_n
		\end{aligned}
	\end{cases}
	\]
	Następnie $i$-tego równania wyliczamy
	$$
	x_i = \frac{b_i - l_i x_{i-1} - u_i x_{i+1}}{d_i}
	$$
	Wyjątkiem są równania pierwsze oraz ostatnie, ponieważ nie istnieją współczynniki odpowiednio $l_1$ oraz $u_n$, stąd:
	$$
	x_1 = \frac{b_1 - u_1 x_2}{d_1}
	$$
	$$
	x_n = \frac{b_n - l_n x_{n-1}}{d_n}
	$$
	
	Wprowadźmy następujące oznaczenia:
	oznaczenia na części rzeczywiste, urojone dla l, d, u, x, b
	
	W $k$-tym kroku obliczamy przybliżenie $x^{(k+1)}$ w taki sposób, że zaczynamy od $x^{(k+1)}_n$, następnie $x^{(k+1)}_{n-1}$, $\dots$, $x^{(k+1)}_1$.
	
	$$
	x^{(k+1)}_n = \frac{b_n - l_n x^{(k)}_{n-1}}{d_n}
	$$
	rozpisać na część rzeczywistą i urojoną
	
	
\end{document}